The laws of motion are formulated in terms of 4 crucial, underlying concepts: the notions of space, time, mass, and force. Some quick review information is provided:

A point $P$ in 3D space can be defined as a vector in any of these ways:
\begin{equation*}
    \vec{r} = r_x \vec{x} + r_y \vec{y} + r_z \vec{z} = (r_x, r_y, r_z) = r_1 {\bm \vec{e}_1} + r_2 {\bm \vec{e}_2} + r_3 {\bm \vec{e}_3} = \sum_{n=1}^{3} r_n e_n.
\end{equation*}

Some vector operations include:
\begin{itemize}
    \item {\bfseries Sum}: $ \vec{r} + \vec{s} = (r_1 + s_1, r_2 + s_2, r_3 + s_3). $ An example is the resultant force administered from the sum of two forces.
    
    \item {\bfseries Scalar multiplication}: $ c\vec{s} = (cr_1, cr_2, cr_3). $ This only affects the magnitude of the resultant vector (unless $c < 0$, then the direction is reversed!). Newton's second law, $\vec{F} = m\vec{a}$, is a common example of scalar multiplication.
    
    \item Scalar product ({\bfseries dot product}): $\vec{r}\cdot\vec{s} = |\vec{r}||\vec{s}|cos(\theta) = r_1 s_1 + r_2 s_2 + r_3 s_3, $ where $|\vec{r}|$ is the magnitude of $\vec{r}$ and $\theta$ is the angle between $\vec{r}$ and $\vec{s}$. A frequent example of the dot product is the definition of work as the product of a force over a displacement, that is, $\vec{W} = \vec{F} \cdot d\vec{r}.$
    
    \item Vector product ({\bfseries cross product}): $ \vec{p} = \vec{r} \times \vec{s} = \begin{vmatrix} \vec{x} & \vec{y} & \vec{z} \\ r_x & r_y & r_z \\ s_x & s_y & s_z \end{vmatrix} = (r_y s_z - r_z s_y)\vec{x} + (r_x s_z - r_z s_x)\vec{y} + (r_x s_y - r_y s_x)\vec{z}. $ $\vec{p}$ is always perpendicular to $\vec{r}$ and $\vec{s}$ by the right hand rule. The magnitude is $|\vec{r}||\vec{s}|sin(\theta)$. An example of this is torque being applied to a point $O$ such that $\vec{\Gamma} = \vec{r} \times \vec{F}.$
\end{itemize}

Another important tool using vectors is {\bfseries differentiation}. Commonly seen in the example of velocity of a particle, $\vec{v}(t)$, is the time derivative of the particles position, $\vec{r}(t)$, that is, $\vec{v}(t) = \frac{d\vec{r}}{dt}$. More formally,
\begin{equation*}
    \frac{d\vec{r}}{dt} = \bfrac{lim}{\Delta t \rightarrow 0} \frac{\Delta \vec{r}}{\Delta t}.
\end{equation*}

\noindent Similarly, for differentiation of vectors,
\begin{equation*}
    \frac{d}{dt}(\vec{r}(t) + \vec{s}(t)) = \frac{\Delta \vec{r}}{\Delta t} + \frac{\Delta \vec{s}}{\Delta t}.
\end{equation*}

\noindent And also for the product of vectors you find the product rule,
\begin{equation*}
    \frac{d}{dt}(f(t) \vec{r}(t)) = f(t) \frac{\Delta \vec{r}}{\Delta t} + \frac{\Delta f}{\Delta t} \vec{r}(t).
\end{equation*}

\noindent It also comes that the differentiation of a vector is equal to the differentiation of each of its constituents such that,
\begin{equation*}
    \frac{d \vec{r}}{dt} = \frac{d \vec{r}_x}{dt}\vec{x} + \frac{d \vec{r}_y}{dt}\vec{y} + \frac{d \vec{r}_z}{dt}\vec{z}.
\end{equation*}

\noindent This is only possible due to the fact that the basic unit vectors are constant in a rectangular coordinate system. Polar coordinates $(r,\theta,\phi)$ are not constant; therefore their derivatives do not share this unique property of Cartesian coordinates.

\vspace{0.5cm} \noindent {\itshape Definition}: Reference Frames: choice in spatial origin and axes to label positions and a choice in temporal origin to measure times. Reference frames can have differing values of spatial coordinates and time, some of which may simplify work in the problem. For example, a problem that asks about a block sliding down an incline may be easier to solve if one axis is pointing along the slope.

{\itshape Non-inertial reference frames} are accelerating relative to their inertial counterparts. In non-inertial frames Newton's laws do not hold. For example, a pendulum in the back of an accelerating truck will seem to have an invisible force acting upon it.

\vspace{0.5cm} \noindent {\itshape Fun Fact}: The mass of an object is characterized by the object's inertia - it's ability to resist being accelerated. The Law of Inertia, Newton's first, states that an object will remain at rest or move at a constant speed in a straight line unless it is acted on by an unbalanced force.
