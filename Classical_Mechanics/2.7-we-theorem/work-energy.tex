
The most basic type of energy is Kinetic Energy ($T = \frac{1}{2}mv^2$), or energy in motion. The Work-Kinetic Energy Theorem is defined as $\Delta T = W$ or $W = dT = \vec{F} \cdot \vec{dr}$, where the second equation is translated to "the work $W$ done by force $F$ in displacement $dr$." Negative work is possible in two obvious cases:

\begin{enumerate}
    \item A reduction in Kinetic Energy
    \item $\vec{F}$ and $\vec{dr}$ are in opposite directions.
\end{enumerate}

Point 2 is best described as "work is path dependent ($P_1 \rightarrow P_2$)," such that $\Delta T = T_2 - T_1 \rightarrow \sum \vec{F} \cdot \vec{dr} = \int_{P_1}^{P_2} \vec{F} \cdot \vec{dr}$.

{\itshape Definition}: {\bfseries The Work-Energy Theorem}: The sum of the work done by $N$ (all) particles from point 1 to point 2 = the change in Kinetic Energy for all $N$ particles.

\begin{equation*}
    \Delta T = T_2 - T_1 = \int_{P_1}^{P_2} \vec{F} \cdot \vec{dr} = \sum^{N}_{i=1} W(1\rightarrow2)
\end{equation*}

% EXAMPLE 2.7 THREE DIFFERENT PATHS... MIA
