
The Euler-Lagrange equation is:

\begin{equation}
    \frac{\partial f}{\partial y} - \frac{d}{dx} \frac{\partial f}{\partial y'} = 0.
    \label{eqn:EulerLagrange}
\end{equation}

The equation is best explained through applications.


% Example of E-L equation
{\exbegin Shortest Path between Two Points}

The length between two points 1 and 2 is given by the equation,

\begin{equation*}
    L = \int^2_1 ds = \int^{x_2}_{x_1} \sqrt{1+y'^2} dx.
\end{equation*}

From equation \ref{eqn:EulerLagrange}, $f(y,y',x)=(1+y'^2)^{1/2}$. By evaluating the two partial derivatives concerned, $\frac{\partial f}{\partial y} = 0$ and $\frac{\partial f}{\partial y'} = \frac{y'}{(1+y'^2)^{1/2}}$. Reducing equation \ref{eqn:EulerLagrange} with the above information yields,

\begin{equation*}
    \frac{d}{dx}\frac{\partial f}{\partial y'} = 0.
\end{equation*}

\noindent This means $\frac{\partial f}{\partial y'}$ is a constant, $C$. $y'^2 = C^2(1+y'^2)$ or $y'^2 =$ constant. Since $y'(x)$ is a constant (call it $m$), we integrate that with respect to x and get $mx + b$, the equation of a line.

\exend

Unlike this example where it was assumed $y=y(x)$, paths can also have two dependent variables with the such as $x=x(u)$ and $y=y(u)$. This yields the equivalent Euler-Lagrange equations:

\begin{gather}
    \frac{\partial f}{\partial x} - \frac{d}{du} \frac{\partial f}{\partial x'} = 0\\
    \frac{\partial f}{\partial y} - \frac{d}{du} \frac{\partial f}{\partial y'} = 0
    \label{eqn:EulerLagrange2}
\end{gather}

The equivalent independent variable ($u$ in equation \ref{eqn:EulerLagrange2}) in Lagrangian mechanics is time $t$. The dependent variable usually specify the position or "configuration" of the system and are denoted by $q_1$, $q_2$, $\dots$, $q_n$. The $n$ here represents {\bfseries generalized coordinates} of the {\bfseries configuration space} such as $n=3$ for 3D Cartesian $(x,y,z)$ or $(r,\theta,\phi)$ spherical coordinates or $N$ particles moving freely in 3D space, $n=3N$.

Lagrange's equations have a couple advantages over the Newtonian formulation. First, it is the same form in any coordinate system. Second, in constrained systems (bead on a wire) the Lagrangian approach eliminates the constraining force, like the normal force on a bead by the wire.

The general definition for the Lagrangian function, $\mathcal{L}$, is $T-U$ where $T$ is the kinetic energy and $U$ is the potential energy. Note, in order to use $T-U$ for the Lagrangian function, the frame must be inertial. The Lagrange equations of motion in Cartesian coordinates is then,

\begin{equation}
    \frac{\partial \mathcal{L}}{\partial q} = \frac{d}{dt} \frac{\partial \mathcal{L}}{\partial \dot{q}}
    \label{eqn:lagrange_2ndLaw}
\end{equation}

\noindent where $q$ is the generalized coordinate ($x$, $y$, or $z$).

Another way to describe this is {\bfseries Hamilton's Principle}: The actual path to which a particle follows between two points 1 and 2 in a given time interval, $t_1$ to $t_2$, is such that the action integral

\begin{equation*}
    S = \int_{t_1}^{t_2} \mathcal{L} dt
\end{equation*}

\noindent is stationary when taken along the actual path. In equation \ref{eqn:lagrange_2ndLaw}, the left-hand side, $\frac{\partial \mathcal{L}}{\partial q}$, is considered the {\bfseries generalized force} and $\frac{\partial \mathcal{L}}{\partial \dot{q}}$ from the right-hand side is considered the {\bfseries generalized momentum}. With this notation, the Lagrange equations take the form: (generalized force) $=$ (rate of change of generalized momentum. Also rewritten as: $F=\frac{d}{dt}p$.

We find that when the Lagrangian is independent of a generalized coordinate $q$ (generalized force [left side in equation \ref{eqn:lagrange_2ndLaw}] $=0$), then the corresponding generalized momentum is conserved. That coordinate $q$ is said to be {\bfseries ignorable} or {\bfseries cyclic}. This is known as Noether's theorem.

The number of coordinates that can be independently varied in a small displacement is the number of {\bfseries degrees of freedom}. When the number of degrees of freedom is equal to the number of generalized coordinates needed to describe the system's configuration, then that system is {\bfseries holonomic}. An example of a non-holonomic system is a ball free to roll along a table (plane). This can be imagined by studying the configuration of the ball (a point on the top of the ball) when following in the path of a right triangle. When reaching the origin, the point on the top of the ball will not be in the same configuration even though the given coordinates $(x,y)$ have returned to the same value.

Some examples for the use of Lagrange's equation is given on page 259 of {\itshape Classical Mechanics} by Taylor. The underlying "rules" (steps 1 and 2 can be done in any order) to solving any problem using Lagrange's equations are:

\begin{enumerate}
    \item Write down the kinetic and potential energies and hence the Lagrangian $\mathcal{L} = T-U$, using a convenient inertial reference frame.
    \item Choose a convenient set of generalized coordinates $q_1$, $\dots$, $q_n$, and find expressions for the original coordinates of step 1 in terms of your chosen generalized coordinates.
    \item Rewrite $\mathcal{L}$ in terms of the generalized coordinates $q_1$, $\dots$, $q_n$ and $\dot{q_1}$, $\dots$, $\dot{q_n}$.
    \item Write down $n$ Lagrange equations.
\end{enumerate}

The Physics GRE will throw you in at any one of these steps.
